\documentclass[nofilelist]{cslthse-msc}
% to show a list of used packages at the end of the document, delete the nofilelist option
%\documentclass{cslthse-msc} 
\usepackage[utf8]{inputenc}
\usepackage[english]{babel}
\usepackage{amsmath}
%\usepackage{amsfonts}
%%\usepackage{amssymb}
\usepackage{amsthm}
%\usepackage{makeidx}
\usepackage{graphicx}
\usepackage[titletoc, header, page]{appendix}
\usepackage{transparent}

% used to display the used files at the end. Select nofilelist as a package option to disable this
\listfiles % initialize

%\geometry{showframe}
%better like this?
\student{Alexander Sandström}{alexander.h.sandstrom@gmail.com}
%\students{Flavius Gruian}{Flavius.Gruian@cs.lth.se}{Camilla Lekebjer}{Camilla.Lekebjer@cs.lth.se}

\thesisnumber{LU-EIT-EX: 2023} % Magic Number! Do not change unless Birger Swahn asks you to do so!
% default is Master. Uncomment the following for "kandidatarbete"/Bachelor's thesis
%\thesistype{Bachelor}{Kandidatarbete}

%\title{Formatting a Master's Thesis}
\title{Drone Camera Control at Varying Latency: A Quality of Experience Study}
\svensktitel{Kamerastyrning på en Drönare vid Varierande Latens: en Studie i Operatörsupplevelse}

%\onelinetitle
%\twolinestitle
\threelinestitle
%\fourlinestitle

%\subtitle{A suitable subtitle}
\company{the Swedish Sea Rescue Society and the Department of Electrical and Information Technology, Lund University}
\supervisors{Fredrik Falkman, \href{mailto:fredrik.falkman@ssrs.se}{\texttt{fredrik.falkman@ssrs.se}}}{William Tärneberg, \href{mailto:william.tarneberg@eit.lth.se}{\texttt{william.tarneberg@eit.lth.se}}}
\examiner{Maria Kihl, \href{mailto:maria.kihl@eit.lth.se}{\texttt{maria.kihl@eit.lth.se}}}

\date{\today}
%\date{January 16, 2015}

\acknowledgements{
If you want to thank people, do it here, on a separate right-hand page. Both the U.S. \textit{acknowledgments} and the British \textit{acknowledgements} spellings are acceptable.

}

\theabstract{
Your abstract should capture, in English, the whole thesis with focus on the problem and solution in 150 words. It should be placed on a separate right-hand page, with an additional \textit{1cm} margin on both left and right. Avoid acronyms, footnotes, and references in the abstract if possible.

Leave a \textit{2cm} vertical space after the abstract and provide a few keywords relevant for your report. Use five to six words, of which at most two should be from the title.
}

\keywords{MSc, BSc, template, report, style, structure}

%% Only used to display font sizes
\makeatletter
\newcommand\thefontsize[1]{{#1 \f@size pt\par}}
\makeatother
%%%%%%%%%%

\begin{document}
\renewcommand{\bibname}{References}

\makefrontmatter
\chapter{Introduction}


This thesis aims to compare a few different ways in which a camera onboard of a drone can be controlled in real-time and evaluate the operator’s experience with the different controls at different latencies. The different controls are to be evaluated through a Quality of Experience (QoE) user experiment, where subjective measurements will be collected from experienced and unexperienced drone operators performing a task with the different controls.

\chapter{Background}

This thesis work is a collaboration between the Swedish Sea Rescue Society and the Department of Electrical and Information Technology at (+the Faculty of Engineering?) Lund University.

\section{Motivation}
Furthermore, SSRS does not have an estimate of the latency of the video-feed from the drone, making it difficult to determine how the camera should be controlled, be it by swiping, pressing they arrow-keys
or setting a location on a map for it to look at. In the emerging field of research around drones and remote control there is a clear correlation between latency and the operator’s experience. However, a comparison of different types of controls and their effect on the operator at different latencies is, as far as the student’s knowledge goes, not documented in the current body of research.


This is a reference to ~\cite{Industry4}.

\section{Swedish Sea Rescue Society}
As can be read on their website \cite{ssrs}, the Swedish Sea Rescue Society (SSRS) is a non-profit organization which was founded at a conference in Stockholm 1906 due to Sweden receiving criticism of its' poor sea rescue. Today, it is a foundation with 40 employees and over 143 000 members, and with their 2400 volunteers manning their different rescue vessels they carry out around 90\% of all sea rescues in Sweden all year around.

\subsection{Innovation}
To make these missions more successful and effective SSRS is innovating with new technologies, where one of them is to have drones that will fly out to the location of a reported emergency and get eyes-on-sight before the rescue boat and its’ crew arrives.

Currently, SSRS has a drone connected to the mobile network that can fly towards and loiter around a set of waypoints given on a map, while also providing video from the gimbal mounted camera. The drone flies with a hardware module running the autopilot software ArduPilot which also supports camera gimbal control. The mechanical gimbal assembly supports three degrees of freedom, however, the software necessary for remotely controlling the camera gimbal is not implemented.

\section{Quality of Experience}
\section{Previous Work}

\chapter{Method}
\section{Hardware}
\section{Experimental Setup}
\section{Experimental Procedure}

\chapter{Results}

\chapter{Discussion}

\chapter{Conclusion}

\chapter{Future Work}


% Should use consistent formatting when it comes to Names ("FirstName LastName", or "F. LastName")
%\printbibliography
\makebibliography{MyMSc}

\begin{appendices}
\chapter{About This Document}
The following environments and tools were used to create this document:
\begin{itemize}
\item operating system: Mac OS X 10.14
\item tex distribution: MacTeX-2014, \url{http://www.tug.org/mactex/}
\item tex editor: Texmaker 5.0.2 for Mac, \url{http://www.xm1math.net/texmaker/} for its XeLaTeX flow (recommended) or pdfLaTeX flow
\item bibtex editor: BibDesk 1.6.3 for Mac, \url{http://bibdesk.sourceforge.net/}
\item fonts \texttt{cslthse-msc.cls} document class): 
\begin{description}
\item{for XeLaTeX}: TeX Gyre Termes, \textsf{TeX Gyre Heros}, \texttt{TeX Gyre Cursor} (installed from the TeXLive 2013)
\item{for pdfLaTeX}: TeX Gyre font packages: tgtermes.sty, tgheros.sty, tgcursor.sty, gtxmath.sty (available through TeXLive 2013) 
\end{description} 
\item picture editor: OmniGraffle Professional 5.4.2
\end{itemize}

\noindent A list of the essential \LaTeX packages needed to compile this document follows (all except \texttt{hyperref} are included in the document class):
\begin{itemize}
\item \texttt{fontspec}, to access local fonts, needs the XeLaTeX flow
\item \texttt{geometry}, for page layout
\item \texttt{titling}, for formatting the title page
\item \texttt{fancyhdr}, for custom headers and footers
\item \texttt{abstract}, for customizing the abstract
\item \texttt{titlesec}, for custom chapters, sections, etc.
\item \texttt{caption}, for custom tables and figure captions
\item \texttt{hyperref}, for producing PDF with hyperlinks
\item \texttt{appendix}, for appendices
\item \texttt{printlen}, for printing text sizes
\item \texttt{textcomp}, for text companion fonts (e.g. bullet)
\item \texttt{pdfpages}, to include the popular science summary page at the end
\end{itemize}

\noindent Other useful packages:
\begin{itemize}
\item \texttt{listings}, for producing code listings with syntax colouring and line numbers
\end{itemize}

\section{Page Size and Margins}
Use A4 paper, with the text margins given in Table \ref{tab:margins}.
\begin{table}[!hbt]
\centering
\caption{Text margins for A4.}\label{tab:margins}
\begin{tabular}{cc}
\hline
\textbf{margin} & \textbf{space} \\
\hline 
top &  3.0cm\\ 

bottom & 3.0cm \\ 
 
left (inside) & 2.5cm \\ 

right (outside) & 2.5cm \\ 

binding offset & 1.0cm \\ 
\hline 
\end{tabular} 
\end{table}

\section{Typeface and Font Sizes}
The fonts to use for the reports are \textbf{TeX Gyre Termes} (a \textbf{Times New Roman} clone) for serif fonts, \textsf{\textbf{TeX Gyre Heros}} (a \textsf{\textbf{Helvetica}} clone) for sans-serif fonts, and finally \texttt{\textbf{TeX Gyre Cursor}} (a \texttt{\textbf{Courier}} clone) as mono-space font. All these fonts are included with the TeXLive 2013 installation. Table \ref{tab:fonts} lists the most important text elements and the associated fonts.
\begin{table}[!hbt]
\caption{Font types, faces and sizes to be used.}\label{tab:fonts}

 \begin{tabular}{ l c c c}
\hline 
\textbf{Element} & \textbf{Face} & \textbf{Size}  & \textbf{\LaTeX size}  \\ 
\hline 
{\huge \textbf{Ch. label}} & {\huge \textbf{serif, bold}} & \thefontsize\huge & \verb+\huge+ \\ 
{\Huge \textbf{Chapter}} & {\Huge \textbf{serif, bold}} & \thefontsize\Huge & \verb+\Huge+ \\ 
{\LARGE \textsf{\textbf{Section}}} & {\Large \textsf{\textbf{sans-serif, bold}}} & \thefontsize\LARGE &  \verb+\LARGE+  \\ 
{\Large \textsf{\textbf{Subsection}}} & {\Large \textsf{\textbf{sans-serif, bold}}} & \thefontsize\Large & \verb+\Large+ \\ 
{\large \textsf{\textbf{Subsubsection}}} & {\Large \textsf{\textbf{sans-serif, bold}}} & \thefontsize\large &  \verb+\large+ \\ 
Body & serif & \thefontsize\normalsize & {\footnotesize \verb+\normalsize+} \\
%{\footnotesize Footnote} & serif  & \thefontsize\footnotesize & {\footnotesize \verb+\footnotesize+} \\
{\footnotesize \textsc{Header}} & {\footnotesize \textsc{serif, SmallCaps}} & \thefontsize\footnotesize &  \\
Footer (page numbers) & serif, regular & \thefontsize\normalsize &  \\
\hline
\textbf{Figure label} & \textbf{serif, bold} & \thefontsize\normalsize & \\
Figure caption & serif, regular & \thefontsize\normalsize & \\
\textsf{In figure} & \textsf{sans-serif} & \textit{any} & \\
\textbf{Table label} & \textbf{serif, bold} & \thefontsize\normalsize & \\
Table caption and text & serif, regular & \thefontsize\normalsize & \\
\texttt{Listings} & \texttt{mono-space} & $\le$ \thefontsize\normalsize & \\
\hline 
\end{tabular} 
\end{table}

\subsection{Headers and Footers}
Note that the page headers are aligned towards the outside of the page (right on the right-hand page, left on the left-hand page) and they contain the section title on the right and the chapter title on the left respectively, in \textsc{SmallCaps}. The footers contain only page numbers on the exterior of the page, aligned right or left depending on the page. The lines used to delimit the headers and footers from the rest of the page are $0.4 pt$ thick, and are as long as the text.

\subsection{Chapters, Sections, Paragraphs}
Chapter, section, subsection, etc. names are all left aligned, and numbered as in this document. 

Chapters always start on the right-hand page, with the label and title separated from the rest of the text by a $0.4 pt$ thick line.

Paragraphs are justified (left and right), using single line spacing. Note that the first paragraph of a chapter, section, etc. is not indented, while the following are indented.

\subsection{Tables}
Table captions should be located above the table, justified, and spaced 2.0cm from left and right (important for very long captions). Tables should be numbered, but the numbering is up to you, and could be, for instance:
\begin{itemize}
\item \textbf{Table X.Y} where X is the chapter number and Y is the table number within that chapter. (This is the default in \LaTeX. More on {\LaTeX} can be found on-line, including whole books, such as \cite{goossens93}.) or
\item \textbf{Table Y} where Y is the table number within the whole report
\end{itemize}
As a recommendation, use regular paragraph text in the tables, bold headings and avoid vertical lines (see Table \ref{tab:fonts}). 

\subsection{Figures}
Figure labels, numbering, and captions should be formed similarly to tables. As a recommendation, use vector graphics in figures (Figure \ref{fig:vectorg}), rather than bitmaps (Figure \ref{fig:rasterg}). Text within figures usually looks better with sans-serif fonts.
\begin{figure}[!hbt]
\centering
\includegraphics[scale=2.5]{examplepic1.pdf} 
\caption{A PDF vector graphics figure. Notice the numbering and placement of the caption. The caption text is indented 2.0cm from both left and right text margin.}\label{fig:vectorg}
\end{figure}

\begin{figure}[!hbt]
\centering
\includegraphics[scale=2.5]{examplepic3.jpg} 
\caption{A JPEG bitmap figure. Notice the bad quality of such an image when scaling it. Sometimes bitmap images are unavoidable, such as for screen dumps.}\label{fig:rasterg}
\end{figure}
For those interested in delving deeper into the design of graphical information display, please refer to books such as \cite{Tufte:1986, few2012show}.

\section{Mathematical Formulae and Equations}
You are free to use in-text equations and formulae, usually in \textit{italic serif} font. For instance: $S = \sum_i a_i$. We recommend using numbered equations when you do need to refer to the specific equations:
\begin{equation}
E = \int_0^{\delta} P(t) dt \quad \longleftrightarrow \quad E = m c^2
\end{equation}
The numbering system for equations should be similar to that used for tables and figures.

\section{References}
Your references should be gathered in a \textbf{References} section, located at the end of the document (before \textbf{Appendices}). We recommend using number style references, ordered as appearing in the document or alphabetically. Have a look at the references in this template in order to figure out the style, fonts and fields. Web references are acceptable (with restraint) as long as you specify the date you accessed the given link \cite{fontspec, CTAN}. You may of course use URLs directly in the document, using mono-space font, i.e. \url{http://cs.lth.se/}.

Make sure you add references as close to the claim as possible~\cite{CTAN}, as shown, not at the end of a whole paragraph. Notice also that there is a space before the reference; best is to use \verb+~\cite{ref}+, to allow for unbreakable spaces. References should not be used after the period marking the end of sentence. Using the reference as follows (end of paragrah, after period) is {\em{strongly discouraged}}, since it says nothing about which specific claim you provide the reference for. \cite{fontspec} 

 

\section{Colours}
As a general rule, all theses are printed in black-and-white, with the exception of selected parts in selected theses that need to display colour images essential to describing the thesis outcome (\textit{computer graphics}, for instance).

A strong requirement is for using \textbf{black text on white background} in your document's main text. Otherwise we do encourage using colours in your figures, or other elements (i.e. the colour marking internal and external references) that would make the document more readable on screen. You may also emphasize table rows, columns, cells, or headers using white text on black background, or black text on light grey background.

Note that the document should look good in black-and-white print. Colours are often rendered using monochrome textures in print, which makes them look different from on screen versions. This means that you should choose your colours wisely, and even opt for black-and-white textures when the distinction between colours is hard to make in print. The best way to check how your document looks, is to print out a copy yourself.

The {\LaTeX}  class defines also a few {\em LTH} standard colours, which you could use in your document for various elements, to adhere to the standard university profile. \\These are: {\color{LTHblue}LTHblue}, {\color{LTHbronze}LTHbronze}, {\color{LTHgreen}LTHgreen}, {\color{LTHpink}LTHpink}, {\color{LTHcyan}LTHcyan}, {\color{LTHgrey}LTHgrey}.

\chapter{Language}

You are strongly encouraged to write your report in English, for two reasons. First, it will improve your use of English language. Second, it will increase visibility for you, the author, as well as for the Department of Computer Science, and for your host company (if any).

However, note that your examiner (and supervisors) are not there to provide you with extensive language feedback. We recommend that you check the language used in your report in several ways:
\begin{description}
\item[Reference books] dedicated to language issues can be very useful. \cite{heffernan2000writing} 
\item[Spelling and grammar checkers] which are usually available in the commonly used text editing environments.
\item[Colleagues and friends] willing to provide feedback your writing.
\item[Studieverkstaden] is a university level workshop, that can help you with language related problems (see \href{http://www.lu.se/studera/livet-som-student/service-och-stod/studieverkstaden}{Studieverkstaden}'s web page).
\item[Websites] useful for detecting language errors or strange expressions, such as
\begin{itemize}
\item \url{http://translate.google.com}
\item \url{http://www.gingersoftware.com/grammarcheck/}
\end{itemize}
\end{description}

\section{Style Elements}
Next, we will just give some rough guidelines for good style in a report written in English. Your supervisor and examiner as well as the aforementioned \textbf{Studieverkstad} might have a different  take on these, so we recommend you follow their advice whenever in doubt. If you want a reference to a short style guide, have a look at \cite{shortstyleguide}.

\subsubsection{Widows and Orphans}

Avoid \textit{widows} and \textit{orphans}, namely words or short lines at the beginning or end of a paragraph, which are left dangling at the top or bottom of a column, separated from the rest of the paragraph.

\subsubsection{Footnotes}

We strongly recommend you avoid footnotes. To quote from \cite{OGSW}, \textit{Footnotes are frequently misused by containing information which should either be placed in the text or excluded altogether. They should be avoided as a general rule and are acceptable only in exceptional cases when incorporation of their content in the text  [is] not possible.} 

\subsubsection{Active vs. Passive Voice}

Generally active voice (\textit{I ate this apple.}) is easier to understand than passive voice (\textit{This apple has been eaten (by me).}) In passive voice sentences the actor carrying out the action is often forgotten, which makes the reader wonder who actually performed the action. In a report is important to be clear about who carried out the work. Therefore we recommend to use active voice, and preferably the plural form \textit{we} instead of \textit{I} (even in single author reports).

\subsubsection{Long and Short Sentences}
A nice brief list of sentence problems and solutions is given in \cite{yalesentences}. Using choppy sentences (too short) is a common problem of many students. The opposite, using too long sentences, occurs less often, in our experience.

\subsubsection{Subject-Predicate Agreement}
A common problem of native Swedish speakers is getting the subject-predicate (verb) agreement right in sentences. Note that a verb must agree in person and number with its subject. As a rough tip, if you have subject ending in \textit{s} (plural), the predicate should not, and the other way around. Hence, \textit{only one s}. Examples follow:
\begin{description}
\item[incorrect] He have to take this road.
\item[correct] He has to take this road.
\end{description}
\begin{description}
\item[incorrect] These words forms a sentence.
\item[correct] These words form a sentence.
\end{description}
\noindent In more complex sentences, getting the agreement right is trickier. A brief guide is given in  the \textit{20 Rules of Subject Verb Agreement} \cite{subjectverb}.

\chapter{Structure}
It is a good idea to discuss the structure of the report with your supervisor rather early in your writing. Given next is a generic structure that is a starting point, but by no means the absolute standard. Your supervisor should provide a better structure for the specific field you are writing your thesis in. Note also that the naming of the chapters is not compulsory, but may be a helpful guideline.
\begin{description}
\item[Introduction] should give the background of your work. Important parts to cover:
\begin{itemize}
\item Give the context of your work, have a short introduction to the area.
\item Define the problem you are solving (or trying to solve).
\item Specify your contributions. What does this particular work/report bring to the research are or to the body of knowledge? How is the work divided between the co-authors? (This part is essential to pinpoint individual work. For theses with two authors, it is compulsory to identify which author has contributed with which part, both with respect to the work and the report.)
\item Describe related work (literature study). Besides listing other work in the area, mention how is it related or relevant to your work. The tradition in some research area is to place this part at the end of the report (check with your supervisor).
\end{itemize}
\item[Approach] should contain a description of your solution(s), with all the theoretical background needed. On occasion this is replaced by a subset or all of the following:
\begin{itemize}
\item \textbf{Method}: describe how you go about solving the problem you defined. Also how do you show/prove that your solution actually works, and how well does it work.
\item \textbf{Theory}: should contain the theoretical background needed to understand your work, if necessary.
\item \textbf{Implementation}: if your work involved building an artefact/implementation, give the details here. Note, that this should not, as a rule, be a chronological description of your efforts, but a view of the result. There is a place for insights and lamentation later on in the report, in the Discussion section.
\end{itemize}
\item[Evaluation] is the part where you present the finds. Depending on the area this part contains a subset or all of the following: 
\begin{itemize}
\item \textbf{Experimental Setup} should describe the details of the method used to evaluate your solution(s)/approach. Sometimes this is already addressed in the \textbf{Method}, sometimes this part replaces \textbf{Method}.
\item \textbf{Results} contains the data (as tables, graphs) obtained via experiments  (benchmarking, polls, interviews). Here you should also describe the individual tables or graphs in text, pointing out interesting outliers and trends.
\item \textbf{Discussion} allows for a longer discussion and interpretation of the results from the evaluation, including extrapolations and/or expected impact. Focus here on a broader view of the results, talking about the relation between the different finds.\footnote{Bad practice is to display graphs in Results and then describe them textually one by one in here. No! Both sections should have some discussion, but one targets individual finds and the other tries to bridge between these adopting a more overarching viewpoint.} This might also be a good place to describe your positive and negative experiences related to the work you carried out.
\end{itemize} 
Occasionally these sections are intermingled, if this allows for a better presentation of your work. However, try to distinguish between measurements or hard data (results) and extrapolations, interpretations, or speculations (discussion).
\item[Conclusions] should summarize your findings and possible improvements or recommendations.
\item[Bibliography] is a must in a scientific report. {\LaTeX} and \texttt{bibtex} offer great support for  handling references and automatically generating bibliographies.
\item[Appendices] should contain lengthy details of the experimental setup, mathematical proofs, code download information, and shorter code snippets. Avoid longer code listings. Source code should rather be made available for download on a website or on-line repository of your choosing.

\end{description}


% display used packages information unless noflielist is used in the cslthse-msc package option
\printfilelist

%make sure we're on even page with the pop-sci
\checkoddpage
\ifoddpage
\else
   \newpage
   \thispagestyle{empty}
   \mbox{ }
\fi
\includepdf[pages={1}]{popsci/popsci.pdf}
\end{appendices}

\end{document}