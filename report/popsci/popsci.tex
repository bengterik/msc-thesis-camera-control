% USE PDFLatex!
% to correctly render Swedish characters

\documentclass{popsci}

\usepackage[utf8]{inputenc}
\usepackage[swedish, english]{babel}

\usepackage{fancyhdr}
\usepackage{titling}
\usepackage{color}
\usepackage{colortbl}
\usepackage{graphicx}
\usepackage{flushend}
\usepackage{lmodern}


% Please specify the presentation date
\presentationsdag{2022-06-07}

% use either of these commands to specify the title of your thesis
\examensarbete{Drones for Sea Rescue: Lab and Field Experiments on Camera Gimbal Control}
% To create a title in two rows, leave examensarbete blank and fill in examensarbeteTwoRows.
%\examensarbeteTwoRows{}{}
\student{Alexander Sandström}
%\students{Magnus Hultin}{Mr X}
\supervisor{William Tärneberg (LTH), Fredrik Falkman (Sjöräddningssällskapet)}
\examiner{Maria Kihl(LTH)}

% Your pop-sci title should be different (more catchy) than your thesis title
\title{Drönare som understöd vid sjöräddning}


\begin{document}

% not more than 4 rows!
%\theabstract{Applikations-specifika processorer är allt mer vanligt för få ut rätt prestanda med så lite resurser som möjligt. Detta arbete har en parametrisk modell för att kunna testa hur mycket resurser som behövs för en specifik applikation.}
\theabstract{Sjöräddningssällskapet (SSRS) driver idag ett projekt där man utforskar möjligheten att använda drönare vid uttryckning till havs. I detta examensarbete har en mjukvara för att styra en drönarkamera implementerats. Den har sedan utvärderats i labb och under en riktig flygning i Göteborgs skärgård.}

{\noindent Vid en uttryckning till havs kan små skillnader i tid vara avgörande för om en utryckning resulterar i en lyckad räddning eller en katastrof. För att kunna ge räddningspersonalen bättre beslutsunderlag i ett så tidigt skede som möjligt driver SSRS ett innovationsprojekt kallat Eyes-On-Scene (EOS), där man undersöker användningen av drönare inom sjöräddning. Drönarna ska vara stationerade längs kusten och kunna skickas ut direkt vid ett larm för att sedan skicka bilder till sjöräddarna på väg till olyckan.

\begin{figure}[!bth] % Use pictures in your pop-vet!
    \includegraphics[trim={0 20cm 0 30cm},width=\columnwidth, clip]{../images/dronepic.png} 
    %\caption{En fin bild}
\end{figure}

Den drönare som tagits fram i projektet är en så kallad fastvinge, som är en snabbare och mer energieffektiv konstruktion än den mer vanliga rotordrönaren. Den har även ett 4G-modem, vilket gör att den kan styras över Internet, oberoende av var operatören befinner sig. 

Fastvingens enda verktyg är en gimbalmonterad kamera, som under flygning kan styras genom att sätta ut GPS-punkter på en karta som kameran sedan riktas mot. Då det inte alltid varit lätt att få exakt det man vill titta på i kameravyn har SSRS varit ute efter ett sätt att styra den manuellt istället.

I mitt examensarbete har jag utvecklat ett gränsnitt för att styra drönarkameran med en joystick. I en labbmiljö har jag sedan undersökt hur olika fördröjningar påverkar en kameraoperatör som använder gränsittet. Mjukvaran har sedan integrerats med EOS-drönarens system för att kunna styra kameran under en riktig flygning, och flera lyckade testflygningar genomfördes. 

Resultaten visar att fördröjning försämrar både operatörens prestation och upplevelse, men att sambanden är olika. Exempelvis tyder resultaten på att vissa nivåer av fördröjningar kan försämra operatörens prestation mer än upplevelsen, och att det omvända kan gälla vid andra nivåer av fördröjning.

Testbädden har potential att användas i vidare studier i operatörsupplevelse, som blir ett allt mer relevant område när tillämpningarna för fjärrstyrning blir allt fler. Mjukvaran för att kontrollera kameran kommer även användas av SSRS i framtida flygningar, där man i år förbereder för skarp utryckning för första gången i projektet.
}
\end{document}
