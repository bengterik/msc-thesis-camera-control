% USE PDFLatex!
% to correctly render Swedish characters

\documentclass{popsci}

\usepackage[utf8]{inputenc}
\usepackage[swedish, english]{babel}

\usepackage{fancyhdr}
\usepackage{titling}
\usepackage{color}
\usepackage{colortbl}
\usepackage{graphicx}
\usepackage{flushend}
\usepackage{lmodern}


% Please specify the presentation date
\presentationsdag{2022-06-07}

% use either of these commands to specify the title of your thesis
\examensarbete{Drones for Sea Rescue: Lab and Field Experiments on Camera Gimbal Control}
% To create a title in two rows, leave examensarbete blank and fill in examensarbeteTwoRows.
%\examensarbeteTwoRows{}{}
\student{Alexander Sandström}
%\students{Magnus Hultin}{Mr X}
\supervisor{William Tärneberg (LTH), Fredrik Falkman (Sjöräddningssällskapet)}
\examiner{Maria Kihl(LTH)}

% Your pop-sci title should be different (more catchy) than your thesis title
\title{Drönare som understöd vid sjöräddning}


\begin{document}

% not more than 4 rows!
%\theabstract{Applikations-specifika processorer är allt mer vanligt för få ut rätt prestanda med så lite resurser som möjligt. Detta arbete har en parametrisk modell för att kunna testa hur mycket resurser som behövs för en specifik applikation.}
\theabstract{Sjöräddningssällskapet (SSRS) driver idag ett projekt där man utforskar möjligheten att använda drönare för att understödja räddningspersonal vid uttryckning till havs. I detta examensarbete har en mjukvara för att styra en drönarkamera implementerats. Den har sedan utvärderats i labb och under riktig flygning.}

{\noindent Vid en uttryckning till havs kan små skillnader i tid vara avgörande för om en utryckning resulterar i en lyckad räddning eller en katastrof. För att kunna ge räddningspersonalen bättre beslutsunderlag i ett så tidigt skede som möjligt driver SSRS ett innovationsprojekt där man undersöker användningen av drönare. Drönarna ska vara stationerade längs kusten och ska kunna flyga ut direkt vid ett larm.

\begin{figure}[!bth] % Use pictures in your pop-vet!
    \includegraphics[trim={0 20cm 0 30cm},width=\columnwidth, clip]{../images/dronepic.png} 
    %\caption{En fin bild}
\end{figure}

Den drönare som tagits fram är en så kallad fastvinge, vilket gör att den både är snabbare och mer energieffektiv än den mer vanliga rotordrönaren. På undersidan av drönaren sitter en kamera monterad på en gimbal som gör det möjligt att vrida kameran i alla tre axlar.

Tidigare kunde SSRS bara styra kameran med hjälp av GPS-punkter som placerades ut på en karta och som drönaren då riktade sin gimbal mot. På grund av störningar kan en GPS-punkt hamna fel, och då behöver man justera punkten genom att sätta ut en ny. SSRS ville se vilka möjligheter som fanns att manuellt justera kameravyn.

I mitt examensarbete har jag skapat ett gränsnitt för att styra drönarkameran med en joystick. För att utvärdera systemet utfördes två experiment, det första i ett labb och den andra under en riktig flygning över Göteborgs skärdgård.

I det första experimentet undersökte jag hur en operatör upplever olika stora fördröjningar i systemet. En högre fördröjning försämrade operatörens förmåga att klara av uppgiften mer, men experimentet visade också att en likvärdig ökning i fördröjning hade större effekt desto högre latensen redan var.

I det andra experimentet jämfördes manuell styrning mot GPS-läget i olika scenarion. Det manuella läget visade sig bättre när drönaren flög rakt och man ville se omkring sig eller följa ett rörligt objekt. För att kolla på en stationär punkt var GPS-läget bättre, men manuell justering behövdes fortfarande för att den skulle titta rätt. 

Arbetets bidrag består till största del av att ha möjliggjort för SSRS att bättre styra kameravyn samt att testbädden som byggts i exjobbet kan användas för vidare studier i operatörsupplevelse.    
}
\end{document}
