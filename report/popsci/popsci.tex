% USE PDFLatex!
% to correctly render Swedish characters

\documentclass{popsci}

\usepackage[utf8]{inputenc}
\usepackage[swedish, english]{babel}

\usepackage{fancyhdr}
\usepackage{titling}
\usepackage{color}
\usepackage{colortbl}
\usepackage{graphicx}
\usepackage{flushend}
\usepackage{lmodern}


% Please specify the presentation date
\presentationsdag{2022-06-07}

% use either of these commands to specify the title of your thesis
\examensarbete{Drones for Sea Rescue: Lab and Field Experiments on Camera Gimbal Control}
% To create a title in two rows, leave examensarbete blank and fill in examensarbeteTwoRows.
%\examensarbeteTwoRows{}{}
\student{Alexander Sandström}
%\students{Magnus Hultin}{Mr X}
\supervisor{William Tärneberg (LTH), Fredrik Falkman (Sjöräddningssällskapet)}
\examiner{Maria Kihl(LTH)}

% Your pop-sci title should be different (more catchy) than your thesis title
\title{Drönare som understöd vid sjöräddning}


\begin{document}

% not more than 4 rows!
%\theabstract{Applikations-specifika processorer är allt mer vanligt för få ut rätt prestanda med så lite resurser som möjligt. Detta arbete har en parametrisk modell för att kunna testa hur mycket resurser som behövs för en specifik applikation.}
\theabstract{Sjöräddningssällskapet driver idag ett projekt där man utforskar möjligheten att använda drönare för att understödja räddningspersonal vid uttryckning till havs. I detta arbete har en mjukvara för att styra kameran ombord på drönaren implementerats och utvärderats under varierande nätverksförhållanden, samt under en riktig flygning.}

{\noindent Vid en uttryckning till havs kan små skillnader i tid vara skillnaden mellan en lyckad räddning och en katastrof. För att kunna ge räddningspersonalen bättre beslutsunderlag i ett så tidigt skeende som möjligt driver Sjöräddningssällskapet ett innovationsprojekt där man undersöker användningen av drönare för att få en tidig bild av olyckor till havs.

Drönaren är en så kallad fastvinge, vilket gör att den både är snabbare och mer energieffektiv än en rotordrönare.

Mitt examensarbete har som en del i detta projekt utforskat möjligheten att manuellt styra kameran med den befintliga mjukvaran ombord på drönaren. Inom arbetet har jag utfört två experiment: det första är ett labbexperiment där försökspersonen fått en uppgift att utföra och där jag sedan tittat på hur bra de klarar av uppgiften när en fördröjning simuleras i systemet. Det andra experimentet gjordes under en flygning i Göteborgs skärgård där jag kollade på när manuell styrning var att föredra över det tidigare.

\begin{figure}[!bth] % Use pictures in your pop-vet!
\includegraphics[trim={0 20cm 0 30cm},width=\columnwidth, clip]{../images/dronepic.png} 
%\caption{En fin bild}
\end{figure}

blablabla
}

% {\noindent För att öka prestandan i dagens processorer finns det vektorenheter och flera kärnor i processorer. Vektorenheten finns till för att kunna utföra en operation på en mängd data samtidigt och flera kärnor gör att man kan utföra fler instruktioner samtidigt. Ofta är processorerna designade för att kunna stödja en mängd olika datorprogram. Detta resulterar i att det blir kompromisser som kan påverka prestandan för vissa program och vara överflödigt för andra. I t.ex. videokameror, mobiltelefoner, medicinsk utrustning, digital kameror och annan inbyggd elektronik, kan man istället använda en processor som saknar vissa funktioner men som istället är mer energieffektiv. Man kan jämföra det med att frakta ett paket med en stor lastbil istället för att använda en mindre bil där samma paketet också skulle få plats.

% I mitt examensarbete har jag skrivit en modell som kan användas för att snabbt designa en processor enligt vissa parametrar. Dessa parametrar väljs utifrån vilket eller vilka program man tänkta köra på den. Vissa program kan t.ex. lättare använda flera kärnor och vissa program kan använda korta eller längre vektorenheter för dess data.

% %För att kunna välja vilken typ av processor som är rätt för den specifika applikationen krävs det ofta att man snabbt kan testa olika prototyper. Att implementera dessa till hårdvara kan ofta vara tidskrävande och ifall det visar sig att implementationen inte klarar dem kraven man ställt för prestanda och energieffektivitet, måste man designa för nya parametrar och mer tid har blivit slösat. %Om den här processen istället kan göras automatiskt utifrån dessa design-parametrar kan man teoretiskt spara en massa tid.
% Modellen testades med olika multimedia program. Den mest beräkningsintensiva och mest upprepande delen av programmen användes. Dessa kallas för kärnor av programmen. Kärnorna som användes var ifrån MPEG och JPEG, som används för bildkomprimering och videokomprimering.

% \begin{figure}[!bth] % Use pictures in your pop-vet!
% \includegraphics[width=\columnwidth]{samplePic.pdf} 
% %\caption{En fin bild}
% \end{figure}
% Resultatet visar att det finns en prestanda vinst jämfört med generella processorer men att detta också ökar resurserna som behövs. Detta trots att den generella processorn har nästan dubbelt så hög klockfrekvens än dem applikations-specifika processorerna. Resultatet visar också att schemaläggning av instruktionerna i programmen spelar en stor roll för att kunna utnyttja resurserna som finns tillgängliga och därmed öka prestandan. Med den schemaläggningen som utnyttjade resurserna bäst var prestandan minst 79\% bättre än den generella processorn.
% }

\end{document}
